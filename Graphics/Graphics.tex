\documentclass[10pt]{beamer}
\usepackage{ragged2e}
\usepackage{tfrupee}  
\usepackage{xcolor,colortbl}
\usepackage{hyperref}
\usepackage{booktabs}
\usepackage{ccicons}
\hypersetup{
colorlinks=true,     
urlcolor= red}
\usetheme[               % right of left position of sidebar (default is right)
  ]{Aalborg}

\usepackage[utf8]{inputenc}
\usepackage[english]{babel}

\usepackage{helvet}
\newcommand{\gray}{\rowcolor[gray]{.05}}
\definecolor{gold}{rgb}{0.85,.66,0}
 \definecolor{khaki}{rgb}{0.941176,0.901961,0.549020}
\definecolor{saffron}{rgb}{0.96, 0.77, 0.19}
\definecolor{sand}{rgb}{0.76, 0.7, 0.5}
\definecolor{ablue}{rgb}{0.94, 0.97, 1.0}
\definecolor{cgreen}{rgb}{0.0, 0.42, 0.24}
\setbeamercolor{bgcolor}{fg=black,bg=ablue}
\setbeamercolor{separation line}{use=structure,bg=structure.fg!20!bg}
\setbeamercolor{block body}{fg=black,bg=ablue}
\setbeamercolor{footlinecolor}{fg=white,bg=sand}
\newenvironment<>{varblock}[2][.9\textwidth]{%
  \setlength{\textwidth}{#1}
  \begin{actionenv}#3%
    \def\insertblocktitle{#2}%
    \par%
    \usebeamertemplate{block begin}}
  {\par%
    \usebeamertemplate{block end}%
  \end{actionenv}}

\newcommand{\ma}{\textbf{Maxima {}}}
\newcommand{\wma}{\textit{WxMaxima {}}}

\setbeamercolor{normal text}{fg=black} 


\newcommand{\chref}[2]{%
  \href{#1}{{\usebeamercolor[bg]{Aalborg}#2}}%
}

\title[The CAS Workshop]
{ Workshop on\\ \textbf{{\huge Computer Algebra System (CAS)}}\vspace{0.2cm}}

\subtitle{\textbf{Session: Graphics with CAS}}  % could also be a conference name

%\date{\today}
\date{10. 04. 2016}

\author[\ccbyncndeu$\,$ P. K. Yadav] % optional, use only with lots of authors
{
	\textbf{P. K. Yadav \& K. Taneja}\\
	\href{mailto:khusboo.taneja@sharda.ac.in}{{\tt khusboo.taneja@sharda.ac.in}}
}


\institute[
Dept.\ of Civil Engineering\\
Dept.\ of Computer Sci. \& Engg.\\
Sharda University\\] 
{
	Department of Civil Engineering\\
	Dept.\ of Computer Sci. \& Engg.\\
}
\pgfdeclareimage[height=0.5cm]{mainlogo}{fig/logo.png} % placed in the upper left/right corner
\logo{\pgfuseimage{mainlogo}}


\pgfdeclareimage[height=1.5cm]{titlepagelogo}{fig/logo.png} % placed on the title page

\titlegraphic{% is placed on the bottom of the title page
  \pgfuseimage{titlepagelogo}

}

\begin{document}
% the titlepage
{\aauwavesbg
\begin{frame}[plain,noframenumbering] % the plain option removes the sidebar and header from the title page
  \titlepage
\end{frame}}


\section{Motivation}
% motivation for creating this theme
\begin{frame}{Motivation}{}
\begin{block}{Graphics with CAS}
\justifying
\vspace{0.5cm}
Interpretation is the final goal of scientific work, and visualization is the most important tool. \\
\vspace{0.2cm}

Visualization in modern scientific study is more widely used in the development of a underlying model of a experimental or physical system.\\
\vspace{0.2cm}

Commercial CAS systems such as Maple\textsuperscript{TM} or 
Mathematica\textsuperscript{TM} provide a comprehensive visualization tool with an intuitive GUI. Maxima on the other hand links to GNUplot, a complete plotting tool but with limited GUI.\\
\vspace{0.2cm}


\vspace{0.2cm}
Here we learn the basics of data and problem visualization with CAS Maxima.
 

 
\end{block}
\end{frame}

\section{2D plot with Maxima}

\begin{frame}[fragile]{The plot2d function}{}
\begin{block}{plot2d()}

The \verb|plot2d| is perhaps the most useful function for plotting in \ma. It come in the following different forms:
\\
\vspace{0.2cm}

\verb|plot2d(plot,xrange,options)| \\
\vspace{0.1cm}
\verb|plot2d([plot 1, ..., plot n], options)|\\
\vspace{0.1cm}
\verb|plot2d([plot 1, ..., plot n], xrange,options)|\\
\vspace{0.2cm}

where, \verb|plot, plot 1,..., plot_n| can be \textcolor{red}{expressions}, \textcolor{red}{function names}, or a \textcolor{red}{list} with the any of these forms: \\
\vspace{0.2cm}
\verb|[[x1, ..., xn], [y1, ..., yn]]|,\\\verb|[discrete, [[x1, y1], ..., [x_n, y_n]]]|, or\\ 
\verb|[parameteric, x_expr, y-expr, t_range]|
\\ 
\vspace{0.2cm}
The syntax for $x$ range is: \verb|[variable, min, max]|.
\\ 

\end{block}
\end{frame}

\subsection{Plot2D ( ) examples}
\begin{frame}[fragile]{The plot2d function}{}
\begin{block}{plot2d() Examples}
Let us start simple by plotting the function: $\color{red}\exp(-t/10)*\cos(10*t)$. \\
\vspace{0.2cm}

\begin{columns}

\column{.5\textwidth}

The default plot for $t$ = 0 to 10

\includegraphics[width=5cm]{fig/fig1}

\column{.5\textwidth}
Plot with \textbf{option}: [y, 0, 1]
\includegraphics[width=5cm]{fig/fig2}

\end{columns}

\end{block}
\end{frame}

\begin{frame}[fragile]{The plot2d function}{}
\begin{block}{plot2d() When y-range is important}
The function that goes to infinite, e.g., $\color{red}\sec(x)$. \\
\vspace{0.2cm}

\begin{columns}

\column{.5\textwidth}

Plot of $\sec(x)$ for $x$ = -2 to 2

\includegraphics[width=5cm]{fig/fig3}

\column{.5\textwidth}
Plot with \textbf{option}: [y, -20, 20]
\includegraphics[width=5cm]{fig/fig4}

\end{columns}

Fixing the range of $y$ provides information when the function $\sec(x)$ will have a finite value.
\end{block}
\end{frame}

\begin{frame}[fragile]{The plot2d function}{}
\begin{block}{plot2d() Plotting different data}

We normally plot 2 or more data in the same plot, and this necessitates the use of \textbf{legend} 

\begin{columns}

\column{.5\textwidth}

we plot: $f(t):=\exp(-t/10)*\cos(10*t)$
and $g(t):=0.5*\exp(-t/10)*\sin(10*t)$

\includegraphics[width=5cm]{fig/fig5}
\vspace{0.3cm}
\column{.5\textwidth}

The legend of the above plot look awful. We replace it with our own.
\includegraphics[width=5cm]{fig/fig6}

\end{columns}

Similarly we can change the \textbf{labels} of \textit{x}-axis and \textit{y}-axis using \verb|[xlabel," " ]|
and  \verb|[ylabel," " ]|.
\end{block}
\end{frame}

\subsection{Discrete data plot}
\begin{frame}[fragile]{The plot2d function}{}
\begin{block}{plot2d() Discrete Data Plots}

Say we have experimental data set pressure (p) changing over time (t), i.e.,
\verb|t = [0,1,2,3,4,5,6]| and \verb|p = [0,1.8,3.5,10.5,12.0,15.5,17.3]|\\

These data can be plotted using \verb|plot2d([discrete, t, p])|.

\includegraphics[width=7cm]{fig/fig7}

We also including the \verb|gri2d| option in the above plot.


\end{block}
\end{frame}

\subsection{Other Plot Options}
\begin{frame}[fragile]{The plot2d function}{}
\begin{block}{plot2d() Plot Options}

The plot can be customized by many more options. Some of the other ones that we have not used are: \verb|[xlabel, ylabel, legend, color, style, point_type,|\\ \verb|nticks, logx, logy, axes, box, plot_realpart]|. 
\vspace{0.2cm}
\begin{columns}
\column{.5\textwidth}
The use of \verb|style|= \verb|[style, points]| and \verb|point_type| = \verb|[point_type, plus]|

\includegraphics[width=5cm]{fig/fig8}


\column{.5\textwidth}

The use of \verb|style|= \verb|[style, linespoints]| and \verb|point_type| = \verb|[point_type, box]|
\includegraphics[width=5cm]{fig/fig9}

\end{columns}

\end{block}
\end{frame}



\begin{frame}[fragile]{The plot2d function}{}
\begin{block}{plot2d() Plot Options}

The plot can be customized in numerous ways and there exist several commands to do that. For more options you should look at \href{http://maxima.sourceforge.net/docs/manual/maxima_100.html#Category_003a-Plotting}{here}. One more plot:
\begin{center}
\includegraphics[width=6cm]{fig/fig10}
\end{center}
The command: \verb|wxplot2d([cos(x),x^2],[x,-1,5],[y,-4,4],|\\\verb|[style,[lines,1,4],[points,0.3,6]]);|

\end{block}
\end{frame}

\section{Plots in Engineering}
\subsection{The Open channel Problem}
\begin{frame}[fragile]{Plots in Engineering}{}
\begin{block}{Revisiting the Open Channel Problem}
Engineering calculation follows analysis. Graphics are used for Engineering Analysis.\\
\vspace{0.2cm}
We revisit the open channel problem once again to find out how graphics can be useful for Engineering Analysis.\\
\vspace{0.2cm}
The specific energy in an open channel is defined as the energy per unit weight is 

$$
E = \frac{v^2}{2g}+y
$$

where E = specific energy, v = flow velocity, g = acceleration of gravity, and y = flow depth

\end{block}
\end{frame}

\begin{frame}[fragile]{Plots in Engineering}{}
\begin{block}{Revisiting the Open Channel Problem}
The flow velocity, in turn, is defined in terms of the unit discharge (or discharge per unit
width), $q$, as $v = q/y$, and replaced into the energy equation as:
$$
E = y+ \frac{q^2}{2gy^2}
$$

when we substitute $q = 5$ m$^2$/s and g=10 m$^2$/s into $E$, we get
$$
E = y+ \frac{1.25}{y^2}
$$

We may re-write the final expression as:

$$
E(y) = y+ \frac{1.25}{y^2}
$$

\end{block}
\end{frame}

\begin{frame}[fragile]{Plots in Engineering}{}
\begin{block}{Revisiting the Open Channel Problem}
$E(y)$ is now defined and can be evaluated for different values of $y$. A plot becomes useful here. We use\\
\vspace{0.2cm}

\verb|plot2d(E(y),[y,0.1,10],[y,0.1,10],[style,[lines,2,2]])|
\centering
\includegraphics[width = 8cm]{fig/fig11}



\end{block}
\end{frame}

\begin{frame}[fragile]{Plots in Engineering}{}
\begin{block}{Revisiting the Open Channel Problem}
Finally, we see how our energy line deviates from the equilibrium line by ploting a $y-y$ line.\\
\vspace{0.1cm}
\scriptsize
\verb|plot2d([E(y),y],[y,0.1,10],[y,0.1,10],[style,[lines,2,2], [lines,1,1]]|,\\\verb|[xlabel,"y(m)"],[ylabel,"E(m)"], [legend, false]|

\begin{center}
\includegraphics[width = 6cm]{fig/fig12}
\end{center}
The plot show that $y>3m$ the Energy will be inaccordance with the equilibrium line. But $y<3m$, $E$ explodes to infinity. 

\end{block}
\end{frame}

\section{Advanced Graphics}
\begin{frame}[fragile]{Advanced graphics}{}
\begin{block}{Statistical Plot}

We now briefly learn about special and advanced graphics that are possible from \ma.\\\vspace{0.2cm}. The Histogram from \ma.
\begin{center}
\includegraphics[width = 8cm]{fig/fig13}
\end{center}

\end{block}
\end{frame}



\begin{frame}[fragile]{Advanced graphics}{}
\begin{block}{Contour Plot}

The contour plot from \ma.
\begin{center}
\includegraphics[width = 10cm]{fig/fig14}
\end{center}

\end{block}
\end{frame}


\begin{frame}[fragile]{Advanced graphics}{}
\begin{block}{The Butterfly}

Get that butterfly curve- parametric curve
\begin{center}
\includegraphics[width = 9cm]{fig/fig15}
\end{center}

\end{block}
\end{frame}

\begin{frame}[fragile]{Advanced graphics}{}
\begin{block}{The 3D plot}

The 3D plot
\begin{center}
\includegraphics[width = 9cm]{fig/fig16}
\end{center}

\end{block}
\end{frame}

\section{Final Thoughts}
\begin{frame}[fragile]{The Final Thoughts}{}
\begin{block}{The helpful links}

We explored a bit about of graphics possibilities with \ma. I am listing few references that you can use to advance yourself.

\begin{enumerate}
\item Maxima manual can be very helpful. Get it from \href{http://maxima.sourceforge.net/docs/manual/maxima.pdf}{here}.
\item A well documented graphics manual can be found at \href{http://www.austromath.at/daten/maxima/zusatz/Graphics_with_Maxima.pdf}{here}.
\item Soon you will realize that \ma contains several additional packages that can be loaded to increase its graphics output. One good documentation can be obtained from \href{http://emp.byui.edu/BrownD/CAS/wxMaxima/Graphs-Graphics-07-wxMax.pdf}{here}
\item Last but not the least, the web-based \ma can be used from \href{http://maxima-online.org/help.html}{here} and the online \ma manual pointing to the graphics functions is at \href{http://maxima.sourceforge.net/docs/manual/maxima_100.html#Category_003a-Plotting}{here}. 


\end{enumerate}


\end{block}
\end{frame}






{\aauwavesbg%
\begin{frame}[plain,noframenumbering]%
  \finalpage{\huge Enjoy the \ma, maths and Good luck with your future works.  \\ \vspace{0.3cm}\textbf{\textcolor{yellow}{Contact Mr. Ruban Sugumar if you need more of \ma}}}
\end{frame}}


\end{document}
